\section{Introduction}
{L'apprentissage profond, et plus spécifiquement les Résaux de Neurones Convolutifs, ont révolutionné le domaine de la vision
par ordinateur, en atteignant aujourd'hui des performances hors norme dans les domaines tels que la reconnaissance d'objet ou 
encore la classification d'images. Bien que ces modèles ont historiquement été conçus pour tourner sur des GPUs de haute 
performance, en priorisant la précision à l'efficacité computationnelle, un changement récent de paradigme change la donne.
En effet, l'émergence du Edge Computing crée un besoin urgent d'adapter ces modèles afin qu'ils soient déployables sur des 
systèmes embarqués.

Bien que le traitement de données en périphérie a un nombre considérable d'avantages tels que la confidentialité ou la baisse de la 
latence, il y a cependant de grandes contraintes qu'on ne peut négliger. Le déploiement sur système embarqué instaure, en effet, des 
contraintes de mémoire, de puissance de calcul ainsi que d'énergie importantes.
Pour répondre à ces contraintes, l'industrie et la communauté scientifique introduisent des solutions innovantes. Ces solutions comprennent 
des architectures neuronales légères, des méthodes de compression des modèles et des systèmes embarqués efficaces.

De nombreux papiers scientifiques se sont penchés sur la question et ont présentés différents benchmarks afin de faciliter la sélection
du couple modèle-machine le plus optimal.}


\subsection{Etat de l'art}


\subsection{Notre contribution}
