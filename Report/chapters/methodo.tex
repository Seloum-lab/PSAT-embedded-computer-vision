\section{Méthodologie}
{Afin d'effectuer au mieux le banchmark, il faut tout d'abord définir précisemment la méthodologie utilisée. Nous allons 
d'abord définir précisemment les différentes métriques que nous allons comparer et détailler la méthode de mesure associée, 
nous allonrs ensuite lister les différents modèles que nous avons décider de comparer entre eux pour enfin spécifier quel aura 
été le support, c'est à dire le hardware.}

\subsection{Métriques}
{Dans le contexte de la vision par ordinateur, il y a un nombre important de métriques à prendre en compte; des métriques qui 
représentent par exemple l'efficacité et/ou la précision. Cependant, en ajoutant la dimension des systèmes embarqués, il y a alors 
encore plus de métriques qui deviennent intéressantes. De plus, certaines métriques deviennent inintéressantes isolées et doivent 
nécessairement être couplés.

La première métrique sur laquelle nous allons nous pencher est celle de l'efficactié, celle des FLOPS. Cette métrique étant particulièrement 
compliqué à mesurer (car le temps d'exécution dépend autant de la mémoire, du parallélisme et du runtime que du nombre d'opérations 
arithmétiques réellement effectuées), nous allons donc nous baser sur les FLOPS théorique du modèle. Cette grandeur représente le coût 
de calcul d'un inférence du point de vue purement algorithmique. Cela calcul théoriquement le nombre d'opérations arithmétiques nécessaires 
étant donné un modèle et la taille de l'entrée. Bien que cette mesure est importante du point de vue du modèle pour estimer son efficacité, elle 
est totalement décorellé du hardware et est donc à prendre avec des pincettes pour le choix du couple machine-modèle.

Ensuite, nous nous intéresserons aux métriques de performance. Nous allons nous pencher tout d'abord sur la latence, c'est à dire la capcité du 
couple modèle-machine à traiter les image le plus vite possible. Cette mértique représente le temps nécessaire au couple pour traiter une image et 
cela se mesure en unité temporelle. Ensuite, nous nous intéresserons aussi au débit. Cela représente la quantité d'information traité par le couple par unité 
de temps. Dans ce cadre, la mesure est faite en FPS (Frame Per Second) et représente le nomre d'image traité par seconde. Bien qu'il y ait une corrélation forte 
entre ces deux grandeurs, il est quand même important de les prendre toutes deux en compte. On peut en effet avoir des modèls qui traitent l'image quasiment 
instantannément, mais en un temps important, tout comme des modèls qui au contraite ont besoin de temps pour traiter l'image, mais peut en traiter plusieurs à 
la fois. Prioriser l'un ou l'autre ou les deux dépendra énormément du cadre d'application et des nécessités de celui-ci.
}